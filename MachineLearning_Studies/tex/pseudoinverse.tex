%%%%%%%%%%%%%%%%%%%%%%%%%%%%%%%%%%%%%%%%%%%%%%%%%%%%%%%%%%%%%%%
%
% Welcome to Overleaf --- just edit your LaTeX on the left,
% and we'll compile it for you on the right. If you give
% someone the link to this page, they can edit at the same
% time. See the help menu above for more info. Enjoy!
%
%%%%%%%%%%%%%%%%%%%%%%%%%%%%%%%%%%%%%%%%%%%%%%%%%%%%%%%%%%%%%%%
\documentclass{article} % Loads settings for the document layout

% Preambel

% The following settings are used for title generation and will show up in the
% main document where the \maketitle command is set.
\title{Pseudo-Inverse Matrix for Data Scientists}
\author{Manish Rai}
\date{2017-08-03}

% Main document

\begin{document} % The document starts here
\maketitle% Creates the titlepage
\pagenumbering{gobble} % Turns off page numbering
\newpage % Starts a new page
\pagenumbering{arabic} % Turns on page numbering

\section{Intuition behind Pseudo-inverse}

The inverse of a matrix depends on it's adjunct (adj(A)) and it's determinant (det(A)), and is given by $A^{-1} = \frac{adj(A)}{|A|}$. This formula shows that for the matrix A had to be a non-singular square matrix for it's inverse to exist.
But we don't always encounter square matrix in our lives and $Moore-Penrose-Pseudoinverse$ of  a matrix gives a fair appromation to the inverse of a matrix.

\section{Applications for a Data Scientists}
\paragraph{}
As a data scientists we come across problems which require computing the inverse of matrices.
We have to deal with problem $Y = XW$, where Y is a column matrix of the response/target variables and X is the N x D marrix with N-datasets and D-dimensions.
\paragraph{}
Each regression problem tries to solve for the weights, W, that minimizes the expected error. And the elegent way tosolve for W is through linear algebra like
\begin{equation}
W_{ML} = X^{-1}Y
\end{equation}
where X inverse is the input matrix and depends on the size of the input dataset. This matrix is not necessarily a square matrix and hence to compute the weights we do need a Pseudoinverse matrix for X which is given as
\begin{equation}
X^{+} = (X^{T}X)^{-1}X^{T}
\end{equation}
\paragraph{}
This was the simplest form of the regression equation. But, a majority of times we preprocess the imnput data (X) or do some feature extraction from it, before feeding it into the model.
This extracted feature can be expressed in terms of the basis function $\phi(x)$.
\paragraph{}
In this case also, we need to calculate the inverse of basis $\phi(x)$ matrix. This is done easily through the pseudoinverse calculation.
\begin{equation}
\phi^{+} = (\phi^{T}\phi)^{-1}\phi^{T}
\end{equation}
\begin{equation}
W_{ML} = (\phi^{T}\phi)^{-1}\phi^{T} Y
\end{equation}
\end{document} % The document ends here
